\documentclass[10pt,twocolumn]{article}
\usepackage[margin=0.8in]{geometry}
\usepackage{graphicx}
\usepackage{caption}
\usepackage{hyperref}
\usepackage{titlesec}
\usepackage{titling}

% ✅ 中文支持
\usepackage{luatexja}
\usepackage{luatexja-fontspec}

% ✅ 设置中文字体
\setmainjfont{WenQuanYi Zen Hei}

\setlength{\droptitle}{-3em}
\setlength{\parskip}{0pt}

\titlespacing{\section}{0pt}{0.8em}{0.4em}
\titlespacing{\subsection}{0pt}{0.6em}{0.3em}

\pretitle{\vspace{-3em} \begin{center}\LARGE\bfseries}
\posttitle{\end{center}\vspace{-1.2em}}
\preauthor{\begin{center}\normalsize}
\postauthor{\end{center}\vspace{-1.5em}}

% 标题和作者信息
\title{\textbf{论文标题}}  % 填写英文原文标题
\author{
    作者1,作者2(如有),等\\   % 最多列出三位作者(英文)
    机构1,机构2(如有)        % 最多列出两个机构(中文)
    % \\\href{https://github.com/your_repo}{\texttt{代码地址:github.com/your\_repo}}  % 如有代码仓库,可启用此行
}

\begin{document}
\maketitle

\noindent \textbf{摘要:} 请用 1-3 句话准确描述研究问题、所用方法、主要贡献与关键结果。

\vspace{0.5em}

\section{引言}
请填写引言部分,包括研究背景、动机与问题定义。

\section{方法}
介绍研究方法,包括模型结构、算法流程、关键技术细节等。可配图说明。
\begin{figure}[h]
    \centering
    \includegraphics[width=0.9\linewidth]{example-image}
    \caption{示意图:模型结构或实验结果展示(请替换)}
\end{figure}

\section{实验设计}
说明实验设置,包括所用数据集、评价指标、对比方法等。可配图/表说明。

\section{实验结果与分析}
报告实验结果,并进行分析与讨论。可以结合图表进行说明。

\begin{table}[h]
\centering
\caption{xxxx}
\begin{tabular}{lcc}
\hline
xx & xxx & xx \\
\hline
xx & xx.x\% & xx.x \\
xx & xx.x\% & xx.x \\
\hline
\end{tabular}
\end{table}

\section{结论}
请总结本文的主要贡献和未来研究方向(如有)。

\end{document}
